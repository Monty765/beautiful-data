

\documentclass{beamer}

\mode<presentation> {


\usetheme{Warsaw}



%\usecolortheme{wolverine}

%\setbeamertemplate{footline}

%\setbeamertemplate{footline}[page number] 

\setbeamertemplate{navigation symbols}{} }

\usepackage{graphicx} 
\usepackage{booktabs} 


\title[Video games' success]{Games' success analysis and visualization}

\author{Ahmed Elkatib, Kaan Ozgunay, Abdulrahman Shabana} 
\institute[CSULA] 
{
California State University Los Angeles \\ 

\medskip

}
\date{\today} 

\begin{document}

\begin{frame}
\titlepage 
\end{frame}

\begin{frame}
\frametitle{Outline} 

\tableofcontents 

\end{frame}


%	PRESENTATION SLIDES

%------------------------------------------------
\section{Data acquisition and storage} 
%------------------------------------------------

\begin{frame}
\frametitle{API's Used to acquire data}
\begin{block}{4Chan :  www.4chan.org }
Using 4chan's API "4ch" and interacting with it with a python wrapper to get all the posts from any given thread and board.
\end{block}

\begin{block}{MetaCritics :  www.metacritic.com}
Using BeautifulSoup to scrape the website to get the release date, Metascore and user score of any given game.
\end{block}

\begin{block}{How the data is stored}
After  manipulating the conversations towards certain games, conversations are saved in text files alongside with Metacritic scores, release dates and game names .
\end{block}
\end{frame}



%------------------------------------------------
\section{Data analysis}
%------------------------------------------------

\begin{frame}
\frametitle{Data Analysis}
\begin {itemize}

\item Filtering previously saved conversations to extract a single score for a single game from the thread

\item Differences between 4chan scores and metacritic scores are calculated and applied to other metacritic scores to get a realistic score for any given game

\item opencv library's k nearest neighbors method is applied to our data, with the attributes game genre, game release date and game average score




\end {itemize}

\end {frame}




%------------------------------------------------

\begin{frame}
\frametitle{Analysis graph}

\begin{figure}[hb]
  \centering
  \includegraphics[width=4in]{knn.png}
  \caption[Close up of ]
   {Given the genre and the release date of a game, the knn method of the opencv library will determine it's nearest neighbors and assign it a score, and plot it}
\end{figure}

\end {frame}
%------------------------------------------------

\begin{frame}[fragile] 
\frametitle{Game's Attributes}
\begin{enumerate}
\item{Genre (color)}
\item{Release Month (X-axis)}
\item{Metacritic's rating (Y-axis)}
    \end{enumerate}

 %\begin{enumerate}
  %\item{blurb}
   % \begin{enumerate}
    %\item{foo}
    %\end{enumerate}


\end{frame}

%\begin{figure}
%\centering
%    \includegraphics[width=0.3\textwidth,natwidth=10,natheight=42]{4chan2.jpg}
%\end{figure}

%\column{.5\textwidth} % Right column and width




%------------------------------------------------


%\begin{frame}
%\frametitle{K-NN algorithm}
%\begin{theorem}[Mass--energy equivalence]
%$E = mc^2$
%\end{theorem}
%\end{frame}

%------------------------------------------------
\section{Data visualization}
%------------------------------------------------
\begin{frame}[fragile] 
\frametitle{Visualization}
\begin{figure}[hb]
  \centering
  \includegraphics[width=4in]{vis.jpg}
 
\end{figure}
\end{frame}
%------------------------------------------------
\begin{frame}
\frametitle{Games over time}

\begin{figure}[hb]
  \centering
  \includegraphics[width=4in]{mona.jpeg}
 
\end{figure}


\end {frame}



%------------------------------------------------
\section{Observations and predictions}
%------------------------------------------------




\begin{frame}[fragile] 
\frametitle{Observations/Predictions}

\end{frame}

%------------------------------------------------
\begin{frame}
\frametitle{Questions}


\begin{figure}
  \centering
  \includegraphics[width=2in]{bigdata.jpg}
 
\end{figure}




\end{frame}



\end{document}
